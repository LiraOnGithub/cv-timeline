%Use cv-timeline class
\documentclass[a4paper]{cv-timeline}
%Use fontawesome5 for icons
\usepackage{fontawesome5}
%Prevent ugly borders around hyperlinks
\hypersetup
	{ pdfborder = {0 0 0}
	}
% Instead of Cantarell you can choose a font you have installed on your machine, or directly link to a font file.
\setmainfont{Cantarell}

\hbadness=99999
\hfuzz=999pt

\definecolor{cv-color}{HTML}{71b8e6}

\begin{document}

	% Everything between \begin{profile} and {\endprofile} goes in the side bar at the left.
	% the second argument, here max.png, is the image you want to use for your CV.
	\begin{profile}{max.png}

		%\profileItem is used for general items, \profileItemWithGap is used when you want to create some vertical empty space.
		%first argument of \profileItem and \profileItemWithGap is the icon, the second the text
		\profileItemWithGap{\faIcon{quote-left}}{Making people happy is fun! It gets me lots of treats! I love treats!}

		\profileItem{\faIcon{user}}{Max (he/him)}
		\profileItem{\faIcon{mobile-alt}}{\href{tel:0032123000000}{+32 123 000000}}
		\profileItem{\faIcon{envelope}}{\href{mailto:goodboy@example.com}{goodboy@example.com}}
		\profileItem{\faIcon{globe}}{\href{https://goodboy.example.com}{goodboy.example.com}}
		\profileItemWithGap{\faIcon{comment}}{Bark (fluent)}
		
		\profileItem{\faIcon{award}}{Stay when asked}
		\profileItem{}{Sit when asked}
		\profileItemWithGap{}{Being cute}

		\profileItem{\faIcon{camera}}{\href{https://www.pexels.com/photo/short-coated-white-and-black-dog-1404727/}{Picture by Krystian Bęben}}
		\profileItem{}{Picture from Pexels}
		\profileItem{\faIcon{font}}{Font by Cantarell}
		\profileItem{\faIcon{font-awesome-flag}}{Icons by Font Awesome}
		\profileItem{\faIcon{wrench}}{Generated by XeLaTeX}

	\end{profile}

	% \timeline consists of two parts: the timeline items and connecting them.
	% There are three items in the timeline:
	%	- \historyItem: contains the mayor items. Arguments are: name, icon, title, date and content
	%	- \historySubItem: same as \historyItem, but indented. They are part of a \historyItem
	%	- \historySeperator: used to separate \historyItems.
	% The content of \historyItem and \historySubItem may contain a \begin{details}...\end{details}.
	% The \detailItems within details work the same way as the \profileItems in the profile part.
	\timeline{
		\historyItem
			{next}
			{\faIcon{star}}
			{Always looking for more fun!}
			{}
				{\begin{details}
					\detailItem {\faIcon{bone}} {and treats!}
				\end{details}}
		\historySeperator
		\historyItem
			{circus}
			{\faIcon{dog}}
			{Circus dog}
			{Since 12/2021}
				{\begin{details}
					\detailItem {\faIcon{laugh-squint}} {Make people laugh!}
					\detailItem {\faIcon{walking}} {Walk on hind legs!}
					\detailItem {\faIcon{square-root-alt}} {Bark four times when clown says "2+2"!}
					\detailItem {\faIcon{bone}} {TREATS!!! \faIcon{heart} \faIcon{heart}}
				\end{details}}
		\historySeperator
		\historyItem
			{police}
			{\faIcon{search-location}}
			{Police dog}
			{06/2021 - 12/2021}
				{\begin{details}
					\detailItem {\faIcon{snowflake}} {Find funny-smelling, white powder!}
					\detailItem {\faIcon{bone}} {Treats!}
				\end{details}}
		\historySeperator
		\historySubItem
			{proof2}
			{\faIcon{paw}}
			{Dog training proof 2}
			{06/2021}
				{\begin{details}
					\detailItem {\faIcon{award}} {Showed people I could do all learned tricks!}
					\detailItem {\faIcon{bone}} {Got lots of treats! Again!}
				\end{details}}
		\historySubItem
			{proof1}
			{\faIcon{paw}}
			{Dog training proof 1}
			{03/2021}
				{\begin{details}
					\detailItem {\faIcon{award}} {Had to show to lots of people how good I can sit still on command!}
					\detailItem {\faIcon{bone}} {Got lots of treats!}
				\end{details}}
		\historyItem
			{training}
			{\faIcon{paw}}
			{Dog training}
			{01/2020 - 06/2021}
				{\begin{details}
					\detailItem {\faIcon{graduation-cap}} {Learned to sit and stand!}
					\detailItem {\faIcon{bone}} {Got treats when I was well-behaved!}
				\end{details}}
	}{
		% Connect all the \historyItems and \historySubItems by their name (first argument of those items)
		% When you have multiple \historySubItems linked to the same \historyItem, the lines can overlap.
		\connect{circus}{next}
		\connect{police}{circus}
		\connect{training}{police}
		% To prevent that overlapping, you can link to an invisible anchor whose name is the name of the \historySubItem followed by "-sp"
		\connect{proof1-sp}{proof2}
		\connect{training}{proof1}
	}

\end{document}
